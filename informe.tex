\documentclass[letterpaper,12pt]{article}

\usepackage[spanish]{babel}
\usepackage[utf8]{inputenc}
\usepackage{graphicx}
\usepackage[top=2cm, left=2cm, right=2cm, bottom=2cm]{geometry}
\linespread{1.5}
\usepackage{amsmath}
\usepackage{mathtools}

\DeclareGraphicsExtensions{.jpg,.pdf,.mps,.png}

\begin{document}

\title{\normalsize{Universidad Simón Bolívar\\ CI-5321 -- Computación Gráfica
II
\\ \textbf{Proyecto 3 -- Informe}\\}}
\author{\normalsize{07-41138 -- Alejandro Machado}\\ \normalsize{07-41279 --
Marilyn Nowacka}}
\date{\normalsize{\today}}
\maketitle

\thispagestyle{empty}
\pagestyle{empty}

El desarrollo de una aplicación WebGL fue una experiencia interesante, al
tratarse de una tecnología experimental que representa el estado del arte en la
computación gráfica accesible a todos a través de la web.

La librería \texttt{three.js} fue de mucha utilidad para realizar la
aplicación: el manejo directo de objetos utilizando \textit{shaders} tal como
hemos visto en el tutorial puede llegar a ser engorroso e innecesariemente
complejo. Esta librería provee una forma de manipular objetos tridimensionales
mediante código javascript fácilmente, dirigida para personas con
experiencia en programación pero no necesariamente expertos en \textit{shaders}
o tecnologías web.

Una de las dificultades que encontramos realizando este proyecto fue relativa a
la carga de modelos OBJ. Fue posible reutilizar un componente del código de
\texttt{three.js} que incluía un \textit{parser} de Wavefront OBJ, pero
encontramos con frecuencia que la carga de modelos arrojaba errores en la
fase de materiales y texturizado. Como los requerimientos del proyecto no incluían el
manejo de materiales (MTL), decidimos cargar solamente la geometría del OBJ y
asignar colores arbitrariamente al modelo.

Otra dificultad fue que al cargar los modelos OBJ se debía ajustar
automáticamente la escala en la que se dibujaba la figura, para poder
representarla en un tamaño razonable en el centro de la pantalla. Para esto
utilizamos las funciones de cálculo de caja cobertora (\textit{bounding boxes})
y escogimos unos valores numéricos de escalamiento que funcionaron
satisfactoriamente en todos los modelos de ejemplo que probamos.
\\

Finalmente, no cumplimos el objetivo de utilizar el esquema de
refinamiento propuesto por Charles Loop.  Para esto debíamos modificar el código 
de \texttt{three.js} que implementa el modelo de Catmull-Clark, lo cual no
logramos realizar. Sí conseguimos utilizar una técnica de teselado para triangularizar las
superficies como preparación para el algoritmo de Loop, pero las subdivisiones
siguen ejecutándose con el algoritmo de Catmull-Clark.
\end{document}
